%%%%%%%%%%%%%%%%%%%%%%%%%%%%%%%%%%%%%%%%%%%%%%%%%%%%%%%%%%%
%% LaTeX 模板,主要针对 A4 纸的中文Paper。
% 文章模板:utf-8编码,A4 纸,10磅,文章类型为article,
% 这里设置UTF8后,下面只需要使用ctex包就能直接用中文
%%%%%%%%%%%%%%%%%%%%%%%%%%%%%%%%%%%%%%%%%%%%%%%%%%%%%%%%%%%%
\documentclass[UTF8,a4paper,10pt]{article}
\usepackage{ctex} % 中文支持
\usepackage{lastpage} % 用于获得最大页数,页眉显示用
\usepackage{geometry} % 用于设置页边距

%%%%%%%%%%%%%%%%%%%%%%%%%%%%%%%%%%%%%%%%%%%%%%%%%%%%%%%%%%%%%%
%定义页边距
\geometry{left=3cm,right=3.8cm,top=2.5cm,bottom=2.5cm}
%定义行间距为1.1倍行距
\renewcommand{\baselinestretch}{1.1}

\title{\textbf{\huge{Latex中文模板}}}
\author{wanzhiping(浙江工业职业技术学院)}

\begin{document}
% 显示title
\maketitle

\begin{center}摘要\end{center}
中文摘要详细内容

\tableofcontents

\section{第一章}
\indent 文献中提到:这是文献一二中提到的内容;
\subsection{列表}
\begin{itemize}  \item 身是菩提树,心如明镜台  \item 时时勤拂拭,勿使惹尘埃.  \item 菩提本无树,明镜亦非台  \item 本来无一物,何处惹尘埃.  \end{itemize}\par
这是一个列表的示范;

\section{插入图片}
pic和tex文件保存在同一路径下  graphfile[30]1.png[picture]\par
可以在tex所在路径建立一个fig 文件夹放图片\par
graphfile[60]fig.1.png[picture]\par

\section{表格}
表格相对比较复杂;\newpage

\section{公式}
公式是LATEX的优势所在; \par
\indent 
$[\alpha + \beta + \varphi + \gamma + \theta * \omega]$
\vspace{3ex}\par
\indent 这是一个兔子;

\section{结束}
 文章结束
\clearpage
\end{document}
