%%%%%%%%%%%%%%%%%%%%%%%%%%%%%%%%%%%%%%%%%%%%%%%%%%%%%%%%%%%
%% LaTeX 模板,主要针对 A4 纸的中文Paper。
% 文章模板:utf-8编码,A4 纸,10磅,文章类型为article,
% 这里设置UTF8后,下面只需要使用ctex包就能直接用中文
%%%%%%%%%%%%%%%%%%%%%%%%%%%%%%%%%%%%%%%%%%%%%%%%%%%%%%%%%%%%
\documentclass[UTF8,a4paper,10pt]{article}
\usepackage{ctex} % 中文支持
\usepackage{fancyhdr}
\usepackage{multicol} % 正文单双栏混排
\usepackage{lastpage} % 用于获得最大页数,页眉显示用
\usepackage{geometry} % 用于设置页边距
\usepackage[subfigure,AllowH]{graphfig} %图片相关

%%%%%%%%%%%%%%%%%%%%%%%%%%%%%%%%%%%%%%%%%%%%%%%%%%%%%%%%%%%%%%
%定义页边距
\geometry{left=3cm,right=3.8cm,top=2.5cm,bottom=2.5cm}
%定义行间距为1.1倍行距
\renewcommand{\baselinestretch}{1.1}

%%%%%%%%%%%%%%%%%%%%%%%%%%%%%%%%%%%%%%%%%%%%%%%%%%%%%%%%%%%%%%
% 标题,作者,通信地址定义
%%%%%%%%%%%%%%%%%%%%%%%%%%%%%%%%%%%%%%%%%%%%%%%%%%%%%%%%%%%%%
% texbf{…}为加粗
% huge{…}等等调节字体的
\title{\textbf{\huge{Latex中文模板}}}
\author{wanzhiping(浙江工业职业技术学院)}

%%%%%%%%%%%%%%%%%%%%%%%%%%%%%%%%%%%%%%%%%%%%%%%%%%%%%%%%%%
% 文章正文
%%%%%%%%%%%%%%%%%%%%%%%%%%%%%%%%%%%%%%%%%%%%%%%%%%%%%%%%%%%%
\begin{document}
%%%%%%%%%%%%
% 此行使文献引用以上标形式显示
\newcommand{\supercite}[1]{\textsuperscript{\cite{#1}}}
% 显示title
\maketitle

\begin{center}摘要\end{center}
中文摘要详细内容

%%%%%%%%%%%%%%%%%%%%%%%%%%%%%%%%%%%%%%%%%%%%%%%%%%%%%%%%%%%%%
% 目录页-------------------------
%%%%%%%%%%%%%%%%%%%%%%%%%%%%%%%%%%%%%%%%%%%%%%%%%%%%%%%%%%%%%
\newpage
\tableofcontents
\newpage

%%%%%%%%%%%%%%%%%%%%%%%%%%%%%%%%%%%%%%%%%%%%%%%%%%%%%%%
% 正文由此开始---------------------
%%%%%%%%%%%%%%%%%%%%%%%%%%%%%%%%%%%%%%%%%%%%%%%%%%%%%%%%%%%
\section{引用文献}
\indent 文献\supercite{ref1,ref2}中提到:这是文献一二中提到的内容;
\subsection{列表}
\begin{itemize}  \item 身是菩提树,心如明镜台  \item 时时勤拂拭,勿使惹尘埃.  \item 菩提本无树,明镜亦非台  \item 本来无一物,何处惹尘埃.  \end{itemize}\par
这是一个列表的示范;

\section{插入图片}
pic和tex文件保存在同一路径下  graphfile[30]1.png[picture]\par
可以在tex所在路径建立一个fig 文件夹放图片\par
graphfile[60]fig.1.png[picture]\par
\begin{Figure}[H]{aaa}[qwd]
\graphfile[20]{./lena.jpg}[picture]
\end{Figure}
\par
上面是一个插图的示范;\par
\begin{Figure}[H]{asdf}[111]  \graphfile[34]{1.png}[picture1]  \graphfile[36]{1.png}[picture2]  \par  \end{Figure}
\par 这是一个双插图的示范;\par

\section{表格}
表格相对比较复杂;\newpage
\vspace{3ex}
\begin{table}[htbp]%图片位置,htbp代表here, top, bottom, page
  \centering%居中
  \caption{This is my first table}\label{table}%label是便于自己查看是那张表的,可不用,注释掉即可,表头一般放在表格上方
  \begin{tabular}{|c|c|c|}
  \hline
  1 & 2 & 3 \\
  \hline
  4 & 5 & 6 \\
  \hline
  7 & 8 & 9 \\
  \hline
\end{tabular}
\end{table}
\vspace{3ex}
\indent 表格示范结束;\par

\section{公式}
公式是LATEX的优势所在; \par
\indent 
\begin{minipage}[h]{0.48\linewidth}
$[\alpha \ge \delta {\rm{ + }}\overline {\eta * \beta } ]$
\end{minipage}\par
\begin{minipage}[h]{0.48\linewidth}
$[\left[ {\begin{array}
{*{20}{c}}{{a_1}}&{{a_2}}&\alpha &\beta \chi &\varphi &\gamma &\eta \theta &{{\zeta _3}}&\xi &\omega
\end{array}}
\right]$
\end{minipage}
\vspace{3ex}\par
\indent 这是一个兔子;

\section{结束}
 文章结束
\newpage
%%%%%%%%%%%%%%%%%%%%%%%%%%%%%%%%%%%%%%%%%%%%%%%%%%%%%%%%%%%%%
% 参考文献
%%%%%%%%%%%%%%%%%%%%%%%%%%%%%%%%%%%%%%%%%%%%%%%%%%%%%%%%%%%%%
\small
\begin{thebibliography}{99}
\setlength{\parskip}{0pt} %段落之间的竖直距离
\bibitem{ref1}吴承恩. 西游记~[M], 明14XX年.
\bibitem{ref2} 玄奘. 大唐西域记学报~[J], 唐6XX年, 1(2): 23-55.
\end{thebibliography}
%%%%%%%%%%%%%%%%%%%%%%%%%%%%%%%%%%%%%%%%%%%%%%%%%%%%%%%%%%%%%
% 文章结束
%%%%%%%%%%%%%%%%%%%%%%%%%%%%%%%%%%%%%%%%%%%%%%%%%%%%%%%%%%%%%
\clearpage
\end{document}
